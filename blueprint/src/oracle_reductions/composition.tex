\section{A Program Logic for Oracle Reductions}

In this section, we describe a program logic for reasoning about oracle reductions. In other words, we define a number of rules or theorems that govern how oracle reductions can be composed to form larger reductions, and how the larger reductions inherit the security properties of the smaller reductions.

The first of these rules is \emph{sequential composition}.

The second is \emph{virtualization}, which allow for reductions on \emph{virtual} or \emph{ghost} values derivable from the actual values in the context.

The third is \emph{substitution (?)}, which allows for substituting a value in the context with another value, followed by a reduction establishing the relationship between the new and old values.

We will also consider weakening / strengthening / framing of predicates on contexts.

\subsection{Changing Relations and Oracles}

% Consequence rule for relations: If we have an oracle reduction for R1 \implies R2, and R1' \implies R1, R2 \implies R2', then we have an oracle reduction for R1' \implies R2' (note: these have to be for the same context)

% Frame rule for relations: R \times R1 \implies R \times R2, given oracle reduction R1 \implies R2

% Lifting shared oracles: if $\mathcal{O}_1 \subset \mathcal{O}_2$, then reductions with $\mathcal{O}_1$ lift to reductions with $\mathcal{O}_2$ with the same security properties

% Consequence rule
\begin{mathpar}
    \inferrule*[lab=Conseq]
    {
        \Psi; \Theta; \Sigma \vdash \{\mathcal{R}_1\} \; \langle\mathcal{P}, \mathcal{V}, \mathcal{E}\rangle^{\mathcal{O}} : \tau \; \{\!\{\mathcal{R}_2; \mathsf{St}; \epsilon\}\!\} \\
        \mathcal{R}_1' \implies \mathcal{R}_1 \\
        \mathcal{R}_2 \implies \mathcal{R}_2'
    }
    {
        \Psi; \Theta; \Sigma \vdash \{\mathcal{R}_1'\} \; \langle\mathcal{P}, \mathcal{V}, \mathcal{E}\rangle^{\mathcal{O}} : \tau \; \{\!\{\mathcal{R}_2'; \mathsf{St}; \epsilon\}\!\}
    }
    \end{mathpar}
    
    % Frame rule
    \begin{mathpar}
    \inferrule*[lab=Frame]
    {
        \Psi; \Theta; \Sigma \vdash \{\mathcal{R}_1\} \; \langle\mathcal{P}, \mathcal{V}, \mathcal{E}\rangle^{\mathcal{O}} : \tau \; \{\!\{\mathcal{R}_2; \mathsf{St}; \epsilon\}\!\}
    }
    {
        \Psi; \Theta; \Sigma \vdash \{\mathcal{R} \times \mathcal{R}_1\} \; \langle\mathcal{P}, \mathcal{V}, \mathcal{E}\rangle^{\mathcal{O}} : \tau \; \{\!\{\mathcal{R} \times \mathcal{R}_2; \mathsf{St}; \epsilon\}\!\}
    }
    \end{mathpar}
    
    % Oracle lifting rule
    \begin{mathpar}
    \inferrule*[lab=Oracle-Lift]
    {
        \Psi; \Theta; \Sigma \vdash \{\mathcal{R}_1\} \; \langle\mathcal{P}, \mathcal{V}, \mathcal{E}\rangle^{\mathcal{O}_1} : \tau \; \{\!\{\mathcal{R}_2; \mathsf{St}; \epsilon\}\!\} \\
        \mathcal{O}_1 \subset \mathcal{O}_2
    }
    {
        \Psi; \Theta; \Sigma \vdash \{\mathcal{R}_1\} \; \langle\mathcal{P}, \mathcal{V}, \mathcal{E}\rangle^{\mathcal{O}_2} : \tau \; \{\!\{\mathcal{R}_2; \mathsf{St}; \epsilon\}\!\}
    }
    \end{mathpar}

TODO: figure out how the state function needs to change for these rules (they are basically the same, but not exactly)

\subsection{Sequential Composition}

The reason why we consider interactive (oracle) reductions at the core of our formalism is that we
can \emph{compose} these reductions to form larger reductions. Equivalently, we can take a complex
\emph{interactive (oracle) proof} (which differs only in that it reduces a relation to the
\emph{trivial} relation that always outputs true) and break it down into a series of smaller
reductions. The advantage of this approach is that we can prove security properties (completeness
and soundness) for each of the smaller reductions, and these properties will automatically transfer
to the larger reductions.

This section is devoted to the composition of interactive (oracle) reductions, and proofs that the
resulting reductions inherit the security properties of the two (or more) constituent reductions.

Sequential composition can be expressed as the folowing rule:

\begin{mathpar}
\inferrule*[lab=Seq-Comp]
{
    \Psi; \Theta; \varSigma \vdash \{\mathcal{R}_1\} \; \langle\mathcal{P}_1, \mathcal{V}_1, \mathcal{E}_1\rangle^{\mathcal{O}} : \tau_1 \; \{\!\{\mathcal{R}_2; \mathsf{St}_1; \epsilon_1\}\!\} \\
    \Psi; (\Theta :: \tau_1) ; \varSigma \vdash \{\mathcal{R}_2\} \; \langle\mathcal{P}_2, \mathcal{V}_2, \mathcal{E}_2\rangle^{\mathcal{O}} : \tau_2 \; \{\!\{\mathcal{R}_3; \mathsf{St}_2; \epsilon_2\}\!\}
}
{
    \Psi; (\Theta :: \tau_1 :: \tau_2) \ ; \varSigma \vdash \{\mathcal{R}_1\} \; \langle\mathcal{P}_1 \circ \mathcal{P}_2, \mathcal{V}_1 \circ \mathcal{V}_2, \mathcal{E}_1 \circ_{\mathcal{V}_2} \mathcal{E}_2\rangle^{\mathcal{O}} : \tau_1 \oplus \tau_2 \; \{\!\{\mathcal{R}_3; \mathsf{St}_1 \oplus \mathsf{St}_2; \epsilon_1 \oplus \epsilon_2\}\!\}
}
\end{mathpar}


% \begin{definition}[Sequential Composition of Protocol Type Signatures]
%     \label{def:protocol_spec_composition}
%     \lean{ProtocolSpec.append}
% \end{definition}

% \begin{definition}[Sequential Composition of Provers]
%     \label{def:prover_composition}
%     \lean{Prover.append}
% \end{definition}

% \begin{definition}[Sequential Composition of Oracle Verifiers]
%     \label{def:oracle_verifier_composition}
%     \lean{OracleVerifier.append}
% \end{definition}

% \begin{definition}[Sequential Composition of Oracle Reductions]
%     \label{def:oracle_reduction_composition}
%     \lean{Reduction.append}
% \end{definition}

\subsection{Virtualization}

Another tool we will repeatedly use is the ability to change the context of an oracle reduction. This is often needed when we want to adapt an oracle reduction in a simple context into one for a more complex context.

See the section on sum-check~\ref{sec:sumcheck} for an example.

\begin{definition}[Mapping into Virtual Context]
    \label{def:virtual_context_mapping}
    In order to apply an oracle reduction on virtual data, we will need to provide a mapping from the current context to the virtual context. This includes:
    \begin{itemize}
        \item A mapping from the current public inputs to the virtual public inputs.
        \item A simulation of the oracle inputs for the virtual context using the public and oracle
        inputs for the current context.
        \item A mapping from the current private inputs to the virtual private inputs.
        \item A simulation of the shared oracle for the virtual context using the shared oracle for
        the current context.
    \end{itemize}
\end{definition}

\begin{definition}[Virtual Oracle Reduction]
    \label{def:virtual_oracle_reduction}
    Given a suitable mapping into a virtual context, we may define an oracle reduction via the following construction:
    \begin{itemize}
        \item The prover first applies the mappings to obtain the virtual context. The verifier does the same, but only for the non-private inputs.
        \item The prover and verifier then run the virtual oracle reduction on the virtual context.
    \end{itemize}
\end{definition}

We will show security properties for this virtualization process. One can see that completeness and soundness are inherited from the completeness and soundness of the virtual oracle reduction. However, (round-by-round) knowledge soundness is more tricky; this is because we must extract back to the witness of the original context from the virtual context.

% virtual-ctx rule of the form: if we have an oracle reduction from R1 to R2 for context prime, a mapping `f` from ctx to ctx prime, and an inverse mapping `g` from witness of ctx prime to witness of ctx, then we have an oracle reduction from (R1 \circ f) to (R2 \circ g) for ctx. The prover & verifier & the state function are composed with `f`, and the extractor needs to be composed with both `f` and `g`.

% Virtualization rule
\begin{mathpar}
\inferrule*[lab=Virtual-Ctx]
{
    \Psi'; \Theta'; \Sigma' \vdash \{\mathcal{R}_1\} \; \langle\mathcal{P}, \mathcal{V}, \mathcal{E}\rangle^{\mathcal{O}} : \tau \; \{\!\{\mathcal{R}_2; \mathsf{St}; \epsilon\}\!\} \\
    f : (\Psi, \Theta, \Sigma) \to (\Psi', \Theta', \Sigma') \\
    g : \Psi' \to \Psi \\
    f.\mathsf{fst} \circ g = \mathsf{id} \\
}
{
    \Psi; \Theta; \Sigma \vdash \{\mathcal{R}_1 \circ f\} \; \langle\mathcal{P} \circ f, \mathcal{V} \circ f, \mathcal{E} \circ (f, g)\rangle^{\mathcal{O}} : \tau \; \{\!\{\mathcal{R}_2 \circ f; \mathsf{St} \circ f; \epsilon\}\!\}
}
\end{mathpar}

\subsection{Substitution}

Finally, we need a transformation / inference rule that allows us to change the message type in a given round of an oracle reduction. In other words, we substitute a value in the round with another value, followed by a reduction establishing the relationship between the new and old values.

% May need multiple rules for different types of substitutions (i.e. whether in context or in protocol type, where replacing oracle / public message with oracle / public / private message, etc.)

Examples include:
\begin{enumerate}
    \item Substituting an oracle input by a public input:
    \begin{itemize}
        \item Often by just revealing the underlying data. This has no change on the prover, and for
        the verifier, this means that any query to the oracle input can be locally computed.
        \item A variant of this is when the oracle input consists of a data along with a proof that
        the data satisfies some predicate. In this case, the verifier needs to additionally check
        that the predicate holds for the substituted data.
        \item Another common substitution is to replace a vector with its Merkle commitment, or a
        polynomial with its polynomial commitment.
    \end{itemize}
    \item Substituting an oracle input by another oracle input, followed by a reduction for each
    oracle query the verifier makes to the old oracle:
    \begin{itemize}
        \item This is also a variant of the previous case, where we do not fully substitute with a
        public input, but do a ``half-substitution'' by substituting with another oracle input. This
        happens e.g. when using a polynomial commitment scheme that is itself based on a vector
        commitment scheme. One can cast protocols like Ligero / Brakedown / FRI / STIR in this
        two-step process.
    \end{itemize}
\end{enumerate}